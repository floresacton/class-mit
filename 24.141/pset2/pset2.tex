\documentclass{article}
\usepackage{amssymb}

\begin{document}

\title{24.141 PSet 2}
\author{Miguel Flores-Acton}
\date{\today}
\maketitle

\section*{Collaborators}
Filipe Abreu
\newline
Tyler Procor

\section*{Q1}
\noindent $"\psi"$ is a sentence letter by itself and can be inferred as just what it
evaluates to.

\noindent $\ulcorner\lnot \psi\urcorner$ is multiple sentence letters, so we have to use corner quotes
to emphasize that we want the value of the sentence letters as an expression, not the raw letters. 

\section*{Q2}
\noindent "Sam is flying to Europe" is a sentence and is an expression.

\noindent "Sam" is "flying to Europe is a sentence" and is an expression.

\noindent "Sam" is "flying to Europe is a sentence and is an expression".

\section*{Q3}
\subsection*{A}
\noindent $\phi \lor \psi$ := $\neg(\neg \phi \land \neg \psi)$
\subsection*{B}
\noindent $\phi \rightarrow \psi$ := $\neg (\phi \land \neg \psi)$
\subsection*{C}
\noindent $\phi \leftrightarrow \psi$ := $\neg(\neg (\phi \land \psi) \land \neg (\neg \phi \land \neg \psi))$

\section*{Q4}
\subsection*{A}
\begin{tabular}{|c|c|c|}
\hline
$\phi$ & $\psi$ & $\phi \uparrow \psi$ \\ \hline
0 & 0 & 1 \\ \hline
0 & 1 & 1 \\ \hline
1 & 0 & 1 \\ \hline
1 & 1 & 0 \\ \hline
\end{tabular}

\subsection*{B}
\noindent $\phi \land \psi$ := $(\phi \uparrow \psi) \uparrow (\phi \uparrow \psi)$

\noindent $\lnot \phi$ := $\phi \uparrow \phi$

\subsection*{C}
This operator is non-homophonic and also a strange truth table. Also, on top of that,
writing some operations feels quite redundant like $\phi \uparrow \phi$ to get a negation.
This would just make it a headache to work with, making sentences longer and less intuitive
to understand their meaning.

\section*{Q5}
\subsection*{A}
\noindent $V_I(\varphi \lor \psi) = V_I(\varphi) + V_I(\psi)$

\noindent $V_I(\varphi \rightarrow \psi) = 1 - (V_I(\varphi) \times (1 - V_I(\psi)))$

\noindent $V_I(\varphi \lor \psi) = (V_I(\varphi) \times V_I(\psi)) + ((1 - V_I(\varphi)) \times (1 - V_I(\psi)))$

\subsection*{B}
Since the semantics above are derived from homophonic operators, then
they can be decomposed to homophonic operators by substitution and are therefore
homophonic in disguise. 


\end{document}
