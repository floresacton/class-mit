\documentclass[12pt]{article}

\usepackage{geometry}
\geometry{verbose,tmargin=1in,bmargin=1in,lmargin=1in,rmargin=1in}
 
\usepackage{amsmath,colonequals,amssymb} %for align* environment and gather*
\usepackage{enumitem} %%Enables control over enumerate and itemize environments
\usepackage{tcolorbox}
 
\newcommand{\corner}[1]{\ulcorner#1\urcorner} %%Corner quotes
\newcommand{\concat}[2]{{#1}{\raisebox{4pt}{\smallfrown}}{#2}} %%Corner quotes
\newcommand{\tuple}[1]{\langle#1\rangle} %%Corner quotes
\renewcommand{\vert}[1]{\lvert#1\rvert}
\newcommand{\set}[1]{\lbrace#1\rbrace} %%Corner quotes
\newcommand{\I}{\mathcal{I}} %%
\newcommand{\J}{\mathcal{J}} %%
\newcommand{\N}{\mathbb{N}} %%
\newcommand{\PL}{\mathcal{L}^{\textsc{pl}}} %%
\newcommand{\FOL}{\mathcal{L}^{\textsc{fol}}} %%
\newcommand{\length}{\texttt{Length}} %%
\newcommand{\comp}{\texttt{Comp}} %%
\newcommand{\V}[1]{\mathcal{V}_{#1}} %%

% Define the \answer{} command
\newcommand{\answer}[1]{%
  \par\noindent
  \begin{tcolorbox}[colback=gray!10, colframe=gray!80, title=Proof]
    #1
  \end{tcolorbox}
}
\newcommand{\factoidbox}[1]{\begin{quote}\framebox{\parbox{\dimexpr\linewidth-3\fboxsep\relax}{#1}}\end{quote}}

\begin{document}

\thispagestyle{empty}

\begin{center}
  \Large Problem Set 4\\[1ex] 
  Due Friday, October 4th by 5pm
  \vspace{.15in}

  \normalsize{(10 points per question. Please scan and upload to Canvas as a PDF)}\\[3ex] 
\end{center}

Question 0: If you worked with up to two classmates, please list their names.\\

\begin{enumerate}
  \item	A finite string whose elements belong to $\set{a,b}$ is an \textsc{a-palindrome} just in case it is a palindrome that has `$a$' as a middle letter.
  \begin{enumerate}[leftmargin=.75in]
    \item[\it Task 1:] Provide a recursive definition of the set of a-palindromes without appealing to the property of being a palindrome as in the rough definition above.
      % NOTE: you can uncomment this block by removing the '%' signs, using this to answer the questions
      
      \answer{
        \begin{enumerate}[label=\arabic*.]
          \item Base case 'a' is an a-palindrome
          \item 
          \item palindrome = letter + palindrome + letter where letter is in \{a,b\}
          \item And any a-palindrome can be built this way
        \end{enumerate}
          
          % \begin{align*}
          %   S &= 1 + 2 + 3 + \cdots + (n-1) + n \\
          %   S &= n + (n-1) + (n-2) + \cdots + 2 + 1 \\
          %   2S &= (1+n) + (2 + (n-1)) + (3 + (n-2)) + \cdots + (n + 1) \\
          %   2S &= (n+1) + (n+1) + (n+1) + \cdots + (n+1) \\
          %   2S &= n(n+1) \\
          %   S &= \frac{n(n+1)}{2}
          % \end{align*}
      
      }
    \item[\it Task 2:] Prove by induction that every a-palindrome has an even number of `$b$'s.
      % NOTE: ditto
      \answer{
        \begin{enumerate}[label=\arabic*.]
          \item Base case: "a" is a palindrome, number of b is even
          \item let k be the length of the palindrome
          \item base case k =1
          \item 
          \item Recursive Step: for any a-palindrome, a...a does not change the number of b's
                so they are still even. If b...b, then number of B's increases by 2 and an even + 2 is still even
          \item Conclusion: an a-palindrome has an even number of b's
        \end{enumerate}
      }
  \end{enumerate}


  \item No wfs of $\PL$ ever contains consecutive atomic formulas (e.g., `$(PP \wedge Q)$').
      Between every atomic formula, there needs to be an operator.
  \item
    Let $\I^+(\varphi) = 1$ for every sentence letter $\varphi$ in $\PL$.
    \begin{enumerate}[leftmargin=.75in]
      \item[\it Task 1:] Show that $\V{\I^+}(\varphi) = 1$ for every wfs $\varphi$ of $\PL$ that does not include negation. 
        % NOTE: ditto
        \answer{
          Repeat here and below...
        }
      \item[\it Task 2:] Show that every contradiction contains negation.
        % NOTE: ditto
        \answer{
          Repeat here and below...
        }
    \end{enumerate}
    
  \item Complete the proof of \textbf{Rule 4 ($\neg$E)} from Chapter 4.

  \item Complete the proof of \textbf{Rule 6 ($\wedge$E)} from Chapter 4.

\end{enumerate}


\end{document}


