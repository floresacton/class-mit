\documentclass[12pt]{article}

\usepackage{geometry}
\geometry{verbose,tmargin=1in,bmargin=1in,lmargin=1in,rmargin=1in}
 
\usepackage{amsmath,colonequals,amssymb} %for align* environment and gather*
\usepackage{enumitem} %%Enables control over enumerate and itemize environments
\usepackage{tcolorbox}
 
\newcommand{\corner}[1]{\ulcorner#1\urcorner} %%Corner quotes
\newcommand{\concat}[2]{{#1}{\raisebox{4pt}{\smallfrown}}{#2}} %%Corner quotes
\newcommand{\tuple}[1]{\langle#1\rangle} %%Corner quotes
\renewcommand{\vert}[1]{\lvert#1\rvert}
\newcommand{\set}[1]{\lbrace#1\rbrace} %%Corner quotes
\newcommand{\I}{\mathcal{I}} %%
\newcommand{\J}{\mathcal{J}} %%
\newcommand{\N}{\mathbb{N}} %%
\newcommand{\PL}{\mathcal{L}^{\textsc{pl}}} %%
\newcommand{\FOL}{\mathcal{L}^{\textsc{fol}}} %%
\newcommand{\FI}{\mathcal{L}^=} %%
\newcommand{\length}{\texttt{Length}} %%
\newcommand{\comp}{\texttt{Comp}} %%
\newcommand{\simp}{\texttt{Simple}} %%
\newcommand{\V}[1]{\mathcal{V}_{#1}} %%
\newcommand{\VV}[2]{\mathcal{V}_{#1}^{#2}} %%
\newcommand{\val}[2]{\mathscr{v}_{#1}^{#2}} %%
\newcommand{\va}[1]{\hat{#1}} %%
\renewcommand{\v}[1]{\mathbf{#1}} %%

% Define the \answer{} command
\newcommand{\answer}[1]{%
  \par\noindent
  \begin{tcolorbox}[colback=gray!10, colframe=gray!80, title=Proof]
    #1
  \end{tcolorbox}
}
\newcommand{\factoidbox}[1]{\begin{quote}\framebox{\parbox{\dimexpr\linewidth-3\fboxsep\relax}{#1}}\end{quote}}

\begin{document}

\thispagestyle{empty}

\begin{center}
  \Large Problem Set 9\\[1ex] 
  Due Friday, November 8th by 5pm
  \vspace{.15in}

  \normalsize{(20 points per question. Please scan and upload to Canvas as a PDF)}\\[3ex] 
\end{center}

If you worked with up to two classmates, please list their names.

\bigskip

\textbf{Note:} Complete the following steps for two of the following three problems given below:

\bigskip

\begin{enumerate}
  \item[(I)] Regiment the argument in $\FI$. (5pts)
  \item[(II)] State whether the argument is valid or invalid. (5pts) 
  \item[(III)] If the argument is invalid, provide a countermodel along with a semantic argument that proves that the argument is invalid. If the argument is valid, provide a semantic argument that proves that it is valid. (10pts)
\end{enumerate}

\bigskip

\begin{enumerate}

  \item	Hesperus is rising. Hesperus is Phosphorus. Therefore Phosphorus is rising.
    % NOTE: you can uncomment this block by removing the '%' signs, using this to answer the questions
    \answer{
      \begin{flushleft}
        (I) \\
        $Rx$: $x$ is rising \\
        Hesperus = $h$ \\
        Phosphorus = $p$ \\
        
        \bigskip
        
        $Rh$ \\
        $h=p$ \\
        --------------- \\
        $Rp$
    \end{flushleft}
      (II) Argument is valid
      \bigskip
      
      (III) Proof:
      \begin{enumerate}[label=\arabic*.]
        \item Let M = $\set{\mathbf{D}, I}$ be a $\FOL$ model where $V_I(h=p) = 1$ and $V_I(Rh) = 1$.
        \item It follows by Lemma 9.2 that $V_I^{\hat{a}}(Rh) = 1$ and $V_I^{\hat{a}}(h=p) = 1$
      \end{enumerate}
      % \begin{enumerate}[label=\arabic*.]
      %   \item Assume a $\FOL$ model $M=\set{\mathbb{D},I}$ where $V_I(Ll)=1$.
      %   \item It follows that $\set{l} \in I(L)$ and $\set{l} \in \mathbb{D}^1$ and $l \in \mathbb{D}$
      %   \item Considering some v.a. $\hat{a}$ where $\hat{a}(x)=l$
      %   then $Ll \rightarrow \lnot Bl$ and therefore
      %   $V_I(Ll) \rightarrow V_I(\lnot Bl)$.
      %   \item Knowing that $V_I(Ll)=1$ in our model, we know that $1 \rightarrow V_I(\lnot Bl)$
      %   \item Based on the semantics of $\rightarrow$ we then know that the previous expression
      %   equates to $V_I(\lnot Bl)$
      %   \item Therefore we can conlude that $\lnot Bl$ is proven by the premises $\square$
      % \end{enumerate}
        
        % \begin{align*}
        %   S &= 1 + 2 + 3 + \cdots + (n-1) + n \\
        %   S &= n + (n-1) + (n-2) + \cdots + 2 + 1 \\
        %   2S &= (1+n) + (2 + (n-1)) + (3 + (n-2)) + \cdots + (n + 1) \\
        %   2S &= (n+1) + (n+1) + (n+1) + \cdots + (n+1) \\
        %   2S &= n(n+1) \\
        %   S &= \frac{n(n+1)}{2}
        % \end{align*}

        % \begin{align*}
        %   \VV{\I}{\va{a}}(\varphi) = 1
        %     ~~\textit{iff}~~  & \VV{\I}{\va{a}}(\qt{\forall}{\gamma} \psi) = 1\\
        %     ~~\textit{iff}~~  & \VV{\I}{\va{e}}(\psi) = 1 \text{ for every $\gamma$-variant } \va{e} \text{ of } \va{a}\\
        %     (\star) ~~\textit{iff}~~  & \VV{\I}{\va{e}}(\psi) = 1 \text{ for every $\gamma$-variant } \va{e} \text{ of } \va{c}\\
        %     ~~\textit{iff}~~  & \VV{\I}{\va{c}}(\qt{\forall}{\gamma} \psi) = 1\\
        %     ~~\textit{iff}~~  & \VV{\I}{\va{c}}(\varphi) = 1.
        % \end{align*}
    }
  \item	If the king of France is bald, then there is a king of France. But there is no king of France. Therefore the king of France is not bald.
  \item Alcibiades and Credo both love Socrates. Therefore Socrates has at least two lovers.
  \answer{
    \begin{flushleft}
      (I) \\
      $Lxy$: $x$ loves $y$ \\
      Alcibiades = $a$ \\
      Credo = $c$ \\
      Socrates = $s$ \\
      
      \bigskip
      
      $Las \land Lcs$ \\
      --------------- \\
      $(\exists x) (\exists y) ((Lxs \land Lys) \land \lnot x=y)$
  \end{flushleft}
    (II) Argument is valid
    \bigskip
    
    (III) Proof:
  }
\end{enumerate}

\bigskip

\textbf{Hints:} Remember...

\begin{enumerate}
  \item Officially the extension of $1$-place predicates is a set of $1$-tuples.
  \item Your semantic proofs ought to appeal to the semantic clauses. Avoid skipping steps in your semantic proofs.
\end{enumerate}



\end{document}
