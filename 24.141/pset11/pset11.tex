\documentclass[12pt]{article}

\usepackage{geometry}
\geometry{verbose,tmargin=1in,bmargin=1in,lmargin=1in,rmargin=1in}
 
\usepackage{amsmath,colonequals,amssymb} %for align* environment and gather*
\usepackage{enumitem} %%Enables control over enumerate and itemize environments
\usepackage{tcolorbox}
 
\newcommand{\corner}[1]{\ulcorner#1\urcorner} %%Corner quotes
\newcommand{\concat}[2]{{#1}{\raisebox{4pt}{\smallfrown}}{#2}} %%Corner quotes
\newcommand{\tuple}[1]{\langle#1\rangle} %%Corner quotes
\renewcommand{\vert}[1]{\lvert#1\rvert}
\newcommand{\set}[1]{\lbrace#1\rbrace} %%Corner quotes
\newcommand{\M}{\mathcal{M}} %%
\newcommand{\I}{\mathcal{I}} %%
\newcommand{\J}{\mathcal{J}} %%
\newcommand{\N}{\mathbb{N}} %%
\newcommand{\PL}{\mathcal{L}^{\textsc{pl}}} %%
\newcommand{\FOL}{\mathcal{L}^{\textsc{fol}}} %%
\newcommand{\comp}{\texttt{Comp}} %%
\newcommand{\V}[1]{\mathcal{V}_{#1}} %%
\newcommand{\VV}[2]{\mathcal{V}_{#1}^{#2}} %%
\newcommand{\val}[2]{\mathfrak{v}_{#1}^{#2}} %%
\newcommand{\va}[1]{\hat{#1}} %%
\renewcommand{\v}[1]{\mathbf{#1}} %%
\newcommand{\unisub}[2]{[#1/#2]}

% Define the \answer{} command
\newcommand{\answer}[1]{%
  \par\noindent
  \begin{tcolorbox}[colback=gray!10, colframe=gray!80, title=Proof]
    #1
  \end{tcolorbox}
}
\newcommand{\factoidbox}[1]{\begin{quote}\framebox{\parbox{\dimexpr\linewidth-3\fboxsep\relax}{#1}}\end{quote}}

\begin{document}

\thispagestyle{empty}

\begin{center}
  \Large Problem Set 8\\[1ex] 
  Due Friday, November 22nd by 5pm
  \vspace{.15in}

  \normalsize{(10 points per question. Please scan and upload to Canvas as a PDF)}\\[3ex] 
\end{center}

Collaborators \\
Felipe Abreu\\
Tyler Proctor\\


\begin{itemize}[leftmargin=1.5in]
  \item[\bf Lemma 11.5] $\VV{\I}{\va{a}}(\varphi) = \VV{\I}{\va{a}}(\varphi\unisub{\beta}{\alpha})$ if $\val{\I}{\va{a}}(\alpha) = \val{\I}{\va{a}}(\beta)$ and $\beta$ is free for $\alpha$ in $\varphi$.
    \begin{enumerate}[leftmargin=-.5in]
      \item Construct an example to show what can go wrong in \textbf{Lemma 11.5} if $\beta$ is not free for $\alpha$ in $\varphi$.
        % NOTE: you can uncomment this block by removing the '%' signs, using this to answer the questions
        \answer{
          Let \( \alpha = x \) be a variable, and let \( \beta = y \) be another variable. We could then create the formula:

          \[
          \phi = \forall x (P(x) \rightarrow Q(x, y))
          \]

          Now, suppose we want to substitute \( \beta = y \) for \( \alpha = x \) in \( \phi \).
          If \( y \) is not free for \( x \) in \( \phi \), then when the variable \( y \) now appears inside the scope
          of the quantifier \( \forall x \) which is not always true.

        }
      \item In the induction step of the proof of \textbf{Lemma 11.5}, fill in the details for the conditional case where $\varphi=(\psi\rightarrow\chi)$.
        \answer{
          Case 4: (\( \psi \to \chi \))
          Assume \( \varphi = \psi \to \chi \), where \( v_I^{\hat{a}}(\alpha) = v_I^{\hat{a}}(\beta) \).  
          Since \( \text{Comp}(\psi \to \chi) = \text{Comp}(\psi) + \text{Comp}(\chi) + 1 \), it follows that \( \text{Comp}(\psi), \text{Comp}(\chi) \leq n \).  
          By the induction hypothesis:
          \[
          v_I^{\hat{a}}(\psi) = v_I^{\hat{a}}(\psi[\beta/\alpha]) \quad \text{and} \quad v_I^{\hat{a}}(\chi) = v_I^{\hat{a}}(\chi[\beta/\alpha]).
          \]
          Using the semantics of implication, we have:
          \[
          v_I^{\hat{a}}(\psi \to \chi) = v_I^{\hat{a}}(\psi[\beta/\alpha] \to \chi[\beta/\alpha]).
          \]
          Thus:
          \[
          v_I^{\hat{a}}(\varphi) = v_I^{\hat{a}}(\varphi[\beta/\alpha]),
          \]
          as desired.
        }
        
        \pagebreak
      \item In the induction step of the proof of \textbf{Lemma 11.5}, fill in the details for the existential case where $\varphi=\exists\gamma\varphi$.
        \answer{
          Case 7: (\( \exists \gamma \varphi \))
          Assume \( \varphi = \exists \gamma \varphi' \), where \( v_I^{\hat{a}}(\alpha) = v_I^{\hat{a}}(\beta) \).  
          If \( \gamma = \alpha \), then \( \alpha \) is not free in \( \varphi \), so \( \varphi = \varphi[\beta/\alpha] \), and:
          \[
          v_I^{\hat{a}}(\varphi) = v_I^{\hat{a}}(\varphi[\beta/\alpha]).
          \]
          Otherwise, assume \( \gamma \neq \alpha \). By the semantics of \( \exists \):
          \[
          v_I^{\hat{a}}(\exists \gamma \varphi') = v_I^{\hat{e}}(\varphi') \quad \text{for some \( \gamma \) variant } \hat{e} \text{ of } \hat{a}.
          \]
          By the induction hypothesis:
          \[
          v_I^{\hat{e}}(\varphi') = v_I^{\hat{e}}(\varphi'[\beta/\alpha]) \quad \text{for all \( \gamma \) variants \( \hat{e} \) of \( \hat{a} \)}.
          \]
          Therefore:
          \[
          v_I^{\hat{a}}(\exists \gamma \varphi') = v_I^{\hat{e}}(\varphi'[\beta/\alpha]) \quad \text{for some \( \gamma \) variant \( \hat{e} \) of \( \hat{a} \)}.
          \]
          Thus:
          \[
          v_I^{\hat{a}}(\varphi) = v_I^{\hat{a}}(\varphi[\beta/\alpha]),
          \]
          as desired.
        }
      \item Explain the role that \textbf{Lemma 11.6} plays in the proof of \textbf{Lemma 11.7}.
        \answer{
          \textbf{Lemma 11.7} uses \textbf{Lemma 11.6} with the assumption that they have the same
          domain $\mathbb{D}$ where $\I(F^n) = \I'(F^n)$ and $\I(a) = \I'(a)$ for every n-place predicate
          $F^n$ and every constant $\alpha \neq \beta$. This relates $\V{\I}^e(\psi) = 1$ to get $\V{\I'}^e(\psi) = 1$
          for every v.a. $e$. This then allows the step
          $\V{\I}(\psi) = \V{\I'}(\psi)$ for all $\psi \in \Gamma$. This also later sets
          $\V{\I'}^c(\varphi) \neq 1$ since we know $\V{\I'}^c(\varphi) \neq 1$ so we can prove \textbf{Lemma 11.7}.
        }
      \item Explain the role that \textbf{Lemma 9.2} plays in the proof of \textbf{Lemma 11.12}.
        \answer{
          Lemma 9.2 lets us introduce a variable assignment $\hat{a}$ from the statement
          $v_I(\varphi[\alpha/\gamma])$ to then use ``$v_I^{\hat{a}}(\varphi[\alpha/\gamma])$ for some v.a.
          in $\hat{a}$ over $\mathbb{D}$''.
        }
    \end{enumerate}
\end{itemize}



\end{document}
