\documentclass{61200}
\usepackage{multicol}

\author{Miguel Flores-Acton}
\problemset{1}

\begin{document}

\problem{1}{None}

\problempart{a}
Ten people is too many people, so it violates policy.

\problempart{b}
This is most likely working too closely together because you are both
typing your solutions together which violates collaboration policy.

\problempart{c}
This also counts as working too closely together and also, your
reciting the solution to your friend which is not allowed.

\problempart{d}
This is not ok, because you cannot help others given that you've already
done and completed the assignment.

\problempart{e}
It is not allowed to share any part of your solution with someone else.

\problempart{f}
Again, it is not allowed to share your answer with someone else.

\problempart{g}
It's not allowed to refrence previos work whether yours or not for the class.

\problempart{h}
As you are just copying an answer and not deriving it yourself,
this is not allowed.

\problempart{i}
Similar to the last situation, you or your peers are not deriving the solution,
just copying it so it is not allowed.

\problempart{j}
Just as it is not allowed to look at the answer from others, it is not ok
to look at answer the answer before completing your assignment.

\problem{2}{None}

\problempart{a}


\begin{multicols}{2}
    \begin{center}
        \begin{tabular}{||c | c | c | c | c||}
            \hline
            A & B & C & B$\rightarrow$C & A$\rightarrow$(B$\rightarrow$C) \\ [0.5ex]
            \hline\hline
            0 & 0 & 0 & 1 & 1 \\
            \hline
            0 & 0 & 1 & 1 & 1 \\
            \hline
            0 & 1 & 0 & 0 & 1 \\
            \hline
            0 & 1 & 1 & 1 & 1 \\
            \hline
            1 & 0 & 0 & 1 & 1 \\
            \hline
            1 & 0 & 1 & 1 & 1 \\
            \hline
            1 & 1 & 0 & 0 & 0 \\
            \hline
            1 & 1 & 1 & 1 & 1 \\
            \hline
        \end{tabular}
    \end{center}
    
    \columnbreak

    \begin{center}
        \begin{tabular}{||c| c | c | c | c | c||}
            \hline
            A & B & C & \scriptsize(A$\rightarrow$B) & \scriptsize(A$\rightarrow$C) & \tiny((A$\rightarrow$B)$\rightarrow$(A$\rightarrow$C))\\ [0.5ex]
            \hline\hline
            0 & 0 & 0 & 1 & 1 & 1 \\
            \hline
            0 & 0 & 1 & 1 & 1 & 1 \\
            \hline
            0 & 1 & 0 & 1 & 1 & 1 \\
            \hline
            0 & 1 & 1 & 1 & 1 & 1 \\
            \hline
            1 & 0 & 0 & 0 & 0 & 1 \\
            \hline
            1 & 0 & 1 & 0 & 1 & 1 \\
            \hline
            1 & 1 & 0 & 1 & 0 & 0 \\
            \hline
            1 & 1 & 1 & 1 & 1 & 1 \\
            \hline
        \end{tabular}
    \end{center}

\end{multicols}



\problempart{b}
If A is false, P := false$\rightarrow$(B$\rightarrow$C), which is always true.
Also, Q := (false$\rightarrow$B)$\rightarrow$(false$\rightarrow$C), which is
true$\rightarrow$true which is true. Therefore P = Q = true when A is false.
\newline
\newline
If A is true, P := true$\rightarrow$(B$\rightarrow$C) which is P := B$\rightarrow$C
\newline
Q := (false$\rightarrow$B)$\rightarrow$(false$\rightarrow$C) which equals
Q := B$\rightarrow$C
\newline
Therefore Q = P = B$\rightarrow$C when A is true.

\problem{3}{David Santana, Riley Davis}

\problempart{a}

\[
\forall n \exists a \exists b \exists c \exists d. n = a^2+b^2+c^2+d^2
\]

\problempart{b}
\[
\forall n \exists a \exists b \exists c. (n > 2) \land (n = 2*a) \land (n = b + c) \land \text{isPrime}(b) \land \text{isPrime}(c)
\]

\problempart{c}
\[
\forall n \exists x \exists y \exists z.(n>2) \land (x^n+y^n=z^n)
\]

\problempart{d}
\[
\forall n \exists a. \text{isPrime}(n) \land (a > n) \land \text{isPrime}(a)
\]
\problempart{e}
\[
\forall n \exists a. (n>1) \land (n < a < 2*n) \land \text{isPrime}(a)
\]
\end{document}

