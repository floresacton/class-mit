\documentclass{article}

\begin{document}

{\tiny
\title{Pset1 - KVL Consistency}
\date{}

\maketitle
}

\section{Approach}
In approaching this problem the first thing that came to
my mind was seeing where Kirchhoff's Voltage Law failed
within the smaller four loops.
\newline
\newline
Calculating the loops counterclockwise comes to:
\newline
Upper left loop: - 4V - (-1V) - 2V + 5V = 0V
\newline
Upper right loop: 4V - 5V + 2V + 3V = 4V
\newline
Bottom right loop: - 2V - (-3V) - 5V + 0V = -4V
\newline
Bottom left loop: 2V - 5V + 4V + (-3V) = -2V

\section{Analyzing}
Seeing that only the upper left loop passes
Kirchhoff's Voltage Law and the other ones fail
it gives a hint to which voltages shouldn't be changed.
If changing any component in the top left loop, then
another component in the same loop would also be changed to
satisfy the law, meaning the bottom left loop would still be
broken. This removes A, C, D, and F from possibly being changed.
\newline
\newline
Looking at the change of voltages through a signle loop,
the two right loops stand out in that they have the same offset.
When thinking more about this, it actually means that both circuits
can be corrected by changing component G because it is counted in
opposite directions when circulating counter clockwise.
\newline
\newline
With this we can recalculate the upper left and right
loops with G as a variable and see that both evaluate to G = -2V:
\newline
Upper right loop: 4V - 5V + (G) + 3V = 0V $\rightarrow$ G = -2V
\newline
Bottom right loop: - (G) - (-3V) - 5V + 0V = 0V $\rightarrow$ G = -2V

\section{Second Change}
Now that we've concluded that component G should change
to -2V, all that is left is the bottom left loop. Here it's
actually possible to change H or K because both of them only
affect this loop.
\newline
\newline
Solving for either we get:
\newline
Bottom left loop: 2V - (H) + 4V + (-3V) = 0 $\rightarrow$ H = 3V
\newline
Bottom left loop: 2V - 5V + (K) + (-3V) = 0 $\rightarrow$ K = 6V

\section{Conclusion}
Final conclusion is that G = -2V for the first component
change and either H = 3V or K = 6V for the second change.

\end{document}