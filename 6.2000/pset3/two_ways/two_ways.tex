\documentclass{article}
\usepackage{amsmath}
\usepackage{graphicx}
\usepackage{float}
\begin{document}

\title{Two Ways}
\author{}
\date{\today}

\maketitle

\section*{1.}
\subsection*{Superposition}
\noindent When only the voltage source is connected and the current source is set to 0A,
we can think of it as the current source is disconnected. This makes just 3 resistors in
series which their equivalent resistance is $R_{eq}=R_1+R_2+R_3$.

\noindent Therefore the current through the resistors is $I=V_s/(R_1+R_2+R_3)$

\noindent Multiplying the current times the resistances, we get that

\noindent $e_1=(V_s\times(R_2+R_3))/(R_1+R_2+R_3)$, and that
$e_2=(V_s\times R_3)/(R_1+R_2+R_3)$

\noindent Now calculating the circuit if $V_s=0$ the equivalent resistance is
$R_{eq}=R_1||(R_2+R_3)$ which expanded makes $1/(1/R_1+1/(R_2+R_3))$.

\noindent Therefore
$e_1=I_s/(1/R_1+1/(R_2+R_3))$, simplifying to

\noindent $e_1=(I_s\times R_1\times R_2\times R_3)/((R_2\times R_3)+R_1)$

\noindent Since $e_2=(e_1\times R_3)/(R_2+R_3)$ we can substitute
$e_1$ to get the equation
$e_2=(I_s\times R_1\times R_2\times {R_3}^2\times (R_2+R_3))/((R_2\times R_3)+R_1)$.

\noindent Combining both of these, we get that\newline
$e_1=(V_s\times(R_2+R_3))/(R_1+R_2+R_3)+(I_s\times R_1\times R_2\times R_3)/((R_2\times R_3)+R_1)$
$e_2=(V_s\times R_3)/(R_1+R_2+R_3)+(I_s\times R_1\times R_2\times {R_3}^2\times (R_2+R_3))/((R_2\times R_3)+R_1)$

\subsection*{Nodal}

With nodal analysis, we can come up with the two equations because we have two unknowns.
$(V_s-e_1)/R_1+I_s+(e_2-e_1)/R_2=0$ and $e_2/R_3=(e_1-e_2)/R_2$

Solving these two equations with some algebra gets us the solutions:\newline
$e_1=(R_2\times V_s+I_s\times R_1\times R_2)/((R_1+R_2)-(R_1\times R_2)/(R_2+R_3))$\newline
$e_2=(R_2\times R_3\times (V_s+I_s\times R_1))/((R_2+R_3)\times((R_2+R_1)-(R_1\times R_3)/(R_2+R_3)))$

\subsection*{Equivalence}

Rearanging the equations from the node and superposition analysis, they are the same.

\section*{2.}

\subsection*{Superposition}
Calculating when only $I_1$ is connected, we first find the equivalent resistance
for the circuit, which is $R_{eq}=75\Omega||150\Omega=50\Omega$

\vspace*{1em}
\noindent That means that $e_1=I_1\times 50$ and then we can treat each pair of resistors as
voltage dividers to calculate $e_2$ and $e_3$

\vspace*{1em}
\noindent Therefore we get\newline
$e_2=(I_1\times 50\times 45)/75=I_1\times 30$ and $e_3=(I_1\times 50)/2=I_1\times 25$

\vspace*{1em}
\noindent Calculating when only $I_2$ is connected, we can first calculate the current divider

\vspace*{1em}
\noindent Current through the bottom will be $I_2\times(105/225)=I_2\times(7/15)$\newline
Current through the top will be $I_2\times(120/225)=I_2\times(8/15)$

\vspace*{1em}
\noindent This means $e_3=75\times I_2\times(7/15)=I_2\times35$ and \newline
$e_2=-45\times I_2\times(7/15)=-I_2\times21$

\vspace*{1em}
\noindent Refrencing $e_1$ from $e_3$ we can calculate\newline
$e_1=I_2\times35-75\times I_2\times(8/15)$ and simplifying\newline
$e_1=-I_2\times5$

\vspace*{1em}
\noindent Adding up the superpositions of $I_1$ and $I_2$ we get:\newline
$e_1=50I_1-5I_2$\newline
$e_2=30I_1-21I_2$\newline
$e_3=25I_1+35I_2$

\subsection*{Nodal}
Calculating with nodal analysis:\newline
$(e_1-e_2)/30=I_2+e_2/45$ and $(e_1-e_3)/75+I_2=e_3/75$ and \newline
$I_1=(e_1-e_2)/30+(e_1-e_3)/75$

\vspace*{1em}
\noindent After solving with algebra:\newline
$e_1=50I_1-5I_2$\newline
$e_2=30I_1-21I_2$\newline
$e_3=25I_1+35I_2$

\subsection*{Equivalence}
These are the same

\end{document}