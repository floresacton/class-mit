\documentclass{article}
\usepackage{amsmath}
\usepackage{graphicx}
\usepackage{float}
\begin{document}

\title{Solving Circuits}
\author{}
\date{\today}

\maketitle

\section*{1}

\noindent Ben is wrong on his first step because he assumes that the voltage
drop across the resistor is 20V, however that is the voltage drop
across both resistors.

\noindent To solve for the current, first find the equivalent resistance
which is $5\Omega$, meaning the current is $\text{20V}/ 5\Omega = \text{4A}$.

\noindent He is correct that by KCL, the current in the $1\Omega$ resistor is also
4A, and therefore $\text{4A}\times1\Omega=\text{4V}$. So $v_1=\text{4V}$.

\section*{2}

\noindent Ben is again wrong on his first step. The resistors are not in parralel,
but in series connecting to the battery. Therefore the equivalent resistance is not
$3\Omega$, but actually $16\Omega$.

\noindent Finishing to solve the circuit we repeat the rest of Ben's steps and get
that the current through the circuit is $\text{9V}/16\Omega=9/16\text{A}$.
Then $v_2=9/16\text{A}*12\Omega=27/4\text{A}$

\section*{3}

\noindent Yay! Ben solved it correctly.

\section*{4}

\noindent Here, Ben just messed up his math with calculating the equivalent resistance
of $6\Omega$ and $10\Omega$. It should be $1/(1/6\Omega+1/10\Omega)=\frac{15}{4}\Omega$.

\noindent Continuing his steps, the total resistance is $2+15/4=23/4$ $\Omega$. Then
$2/(23/4)=8/23\text{ A}$. Then $8/23\text{ A}\times15/4\text{ }\Omega=30/23$ V.

\noindent Then finally,
the voltage drop $v_4=\text{8V}+30/23\text{ V}$

\section*{5}

\noindent Here, Ben is assuming that the 12V source is grounded, but it's not, so we actually
need to perform nodal analysis.

\noindent Before preforming nodal analysis, we should probably make our life easier by
swapping the 12V source with the $4\Omega$ resistor, to create a Thevanin equivalent
circuit of those two components. This makes it where there are no floating voltage sources
and makes our lives easier.

\noindent Now our equation for $e_1$ becomes $(9-e_1)+(12-e_1)/4=e_1/2$

\noindent Solving this for $e_1$ with algebra comes to $e_1=48/7\text{ V}$

\noindent Finally solving for $v_5$, we get $v_5=9\text{V}-48/7\text{ V}=15/7\text{ V}$

\end{document}