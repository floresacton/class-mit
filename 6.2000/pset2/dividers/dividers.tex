\documentclass{article}
\usepackage{amsmath}
\usepackage{graphicx}
\usepackage{float}
\begin{document}

\title{Mo Dividers}
\author{}
\date{\today}

\maketitle

\section{}

Firstly we know that $i_1$ will always be 2A because of the current source.
\newline
Next we can solve for $i_2$ by finding the equivalent resistance of the three resistors
in parrallel, then calculating the current divider. $6\Omega$ $||$ $12\Omega$ =
$4\Omega$ and $4\Omega$ $||$ $4\Omega$ = $2\Omega$. Therefore the current divider means
that $i_2$ = 2A*($2\Omega$/$12\Omega$) = $1/3$A
\newline
\newline
Since we have no voltage source and know that the resistance across $v$ is $20\Omega$,
we have to calculate $v$ using V = I$\times$R: V = (-2A)*($20\Omega$) = -40V

\section{}

We already know that $i_1$ has to be 2A because of the current source. However
for calculating $v$ and $i_2$ we can now use the voltage source and resistor divider
logic and ignore the current source. From earlier, the equivalent resistance of the three
resistors is $2\Omega$ which means the total resistance across the section is
$2\Omega + 20\Omega = 22\Omega$. This means the current through the equivalent resistor
is 90V/$22\Omega$. Based on the current divider principle, $i_2$ should then be
(90/22)A*($2\Omega$/$12\Omega$)=(15/22)A.
\newline
$v$ can then be calculated by the resistor
devider equation of 90V*($20\Omega/22\Omega$) = (900/11)V


\end{document}